\section{\inGerman{Benutzen von RetroShare}
\inEnglish{Using RetroShare}}
\subsection{\inGerman{Der erste Start}
\inEnglish{The first start}}
%
\inGerman{Beim ersten Start wird RetroShare dir die Möglichkeit bieten, einen neuen GnuPG-Schlüsselpaar zu erzeugen. 
Falls du bereits einen GnuPG-Schlüssel hast, z.B. für Email-Verschlüsselung, kannst du auch diesen verwenden.
Dabei muß erwähnt werden das nur \emph{RSA}-Schlüssel verwendbar sind, keine \emph{DSA}-Schl"ussel.}
\inEnglish{On the first start, RetroShare will give you the possibility to create a new Public/Private Keypair.
If you already have a PGP key (e.g. for email encryption), you can use this one by importing it.
Unfortunately RetroShare can use only RSA keys at the moment, but not DSA keys.}

\inGerman{Wenn RetroShare ein PGP-Schlüsselpaar erstellt, benötigt es einen (Nick-)Namen, eine beliebige Email-Adresse (wird nicht auf Gültigkeit geprüft und auch später nicht benötigt), ein Passwort und einen sogenannten \glqq Ort\grqq . Bei der eMail-Adresse steht zwar \emph{optional}, aber RetroShare kann ohne eine Adresse keinen Schlüssel von OpenPGP erstellen lassen.}
\inEnglish{When RetroShare creates a PGP-keypair, it'll need a (nick)name, a free chosen email (will not be checked and visible to all your friends), a passwort and finally a location. The email field has the attribute optional, but can't leave it empty, otherwise OpenPGP can't create a key.}

\inGerman{Der Ort dient dazu, das Menschen z.B. einen Laptop und einen PC unterscheiden können.
Denn \glqq Laptop\grqq oder \glqq Server\grqq lassen sich leichter unterscheiden als rein zufällige Zahlenkollonen wie\\ \glqq 2ddf5cefd2517fb41fd46b5bbb7ce36d\grqq.
RetroShare generiert mit Hilfe des OpenPGP-Key für jede Installation einen neuen SSL-Key, anhand der die einzelnen Installationen unterschieden werden und die Kommunikation verschlüsselt wird.
Bei sämtlichen Installationen kann man denselben GnuPG-Key verwenden.}
\inEnglish{The location is useful for people with more computers, e.g a tower and a laptop. This is a convienient feature to give multiple locations meaningful names, instead of using the random SSLID like 
"2ddf5cefd2517fb41fd46b5bbb7ce36d".}


\inGerman{Ein RetroShare-Zertifikat sieht folgendermaßen aus (dieses hier ist kein real benutztes):}
\inGerman{A RetroShare-Certificate looks like this (this is not a valid one!):}
\begin{lstlisting}
 -----BEGIN PGP PUBLIC KEY BLOCK-----
Version: GnuPG v1.4.11 (GNU/Linux)

mQENBE9aG9gBCADAA09oJZJzUSU7n1udB7o34L9orvp5ckZkB0I/yq4V9446mOio
rSfaVDQnFjETV0NbrM51RIvenUHP75Jzq4/QXcyFE5B+poMg3pVrVUXAm31HZd2S
tRxprxVFpPTplLutNv79WN+NSih8fBFjc8buAKquff6uue9tvX0mj47NNzp6iSR6
+Ae1vzEI0IIw+wKxsdsMcZdPjbJXYRD5eaYnxbhjdJcpr3Sx73XTbOONs7DoNyxr
gEURfV4PIVUWF/zmZmKgp5Gtko4k4k+LRKaTNHZ7rEvPvDqv8O2lXrAmdXG81kuC
XH6K4/ZrLQzltvhqWMzmxgD9OuJunRv2X2bjABEBAAG0M2tsYXVzIChnZW5lcmF0
ZWQgYnkgUmV0cm9zaGFyZSkgPGtsYXVzQGV4YW1wbGUuY29tPokBOAQTAQIAIgUC
T1ob2AIbLwYLCQgHAwIGFQgCCQoLBBYCAwECHgECF4AACgkQnDxC1maIY6nF8Qf/
ZZZPni5VYKWKaN+j5rIkjWoDtySeE3iCOCAJtyjiXVGsRWKaSFPZSiL+8VZl6OPY
N6oAJnWjyDDf2Ql/QUJKfKpcqpe0SowxMGuDiOHkwCp+Ac6g1tWAi+zRPwP93Af1
nI7dNa1TZGmjVJSIvU90JTUM7gCx7vpJf59UZqDatggLxzaeNo2ryXjD+/npRCqe
F/kKRPp3/Oac1IIsYU8JK37uzIQJv3Nv3yBkP73OoZhEq7+g2tw9xsFfZn8skbpR
eXutwvuVFJbGGJLCWikN/DgGl21RaobeiFt109T6LMbsBkIHq9paPVbu7yTgL2TU
iLka9sOAwIjZX0tM5DpzIg==
=7I5x
-----END PGP PUBLIC KEY BLOCK-----
--SSLID--2ddf5cefd2517fb41fd46b5bbb7ce36d;--LOCATION--laptop;
--LOCAL--192.168.2.103:23822;--EXT--93.61.21.14:7812;
--DYNDNS--<subdomain>.<domain>.com;
\end{lstlisting}

\inGerman{Es gibt mittlerweilen ein neues, f"ur den Computer leichter zu lesenden Zertifikats-Format, dass dann etwa so aussieht:}
\inEnglish{There is a new, more robust format for certificates, it looks like this:}
\begin{lstlisting}
Af8AAAJbxsBNBFBq+SABCADfBXgKPDeC4Q6gnOaywnx9XTRcdQQYGvbWAOcygGDx
P7UC9FJ2v8LxtXd6QOjxsexXjGCrey78pPxDgm+iRCG0FGBeLpGBTouamvwQ7uUz
hLY8IGyjy4oDxwXgvVF/0x0WBi1i/haYJi8qXk9/Ll9cDXTSBKfqH2ACFzWum4mt
7klubMhsL80QZVeAeaeI2r6zbgYqaw7Xc1kNhYQDbfUU2m1urzaJ9gOT+MzVi97h
ukjUrE9SuIfrEoqIyL67sflfQyBwYEJm+X2N7pW4CwcnJWsHPI+Fe4POLgrH17bM
dZkIFdN5EJl/1MT3FYLj/zx5c4Fgocmhi3s1xUWz5mbJABEBAAHNJ3Rlc3QxIChH
ZW5lcmF0ZWQgYnkgUmV0cm9TaGFyZSkgPHRlc3QxPsLAXwQTAQIAEwUCUGr5IAkQ
4XzDoqr5prgCGQEAAJ7HCADeRHF2AIUpT0w9/W6+r3e8HiCHaXNsFMcUgrarWl7h
MS0HfmLgVtaku2q17zcj+yS6QbDBGP2j/3+/OpJyQ1JPCBnvhEE3pbUm8Aoe4ZjI
jZofcyGA8fR9ICsCVXGqZE7IiNLuklNcwzIbpWt4+tmgQDO5x9D27ch2QEYisbT9
WZHAxfgW4QPzdKTJiqLxW3xIJqI/tP/y6XByOX/NR57HTXSYcCwE2JTDfuaO2Ki8
RROqu8XXQj/0xPf8QI8osxl2rH3LRx/c2CooPIQIcX64vqaVaol4P7FnTC7czUq+
xdlS/d9gBPkqsbl0j16P56wBmu02NfEBQlxEgwAXiJHKAgZP0FxzBP4DBsCoAncE
/gQABgh0ZXN0MWxvYwUQCCw1pLwJuX1PTasINX94pQ==
\end{lstlisting}

\inGerman{Beide Formate sind exakt identisch und enthalten dieselbe Information, insbesondere enth"alt auch das neue Format die IP.}
\inEnglish{Both Certificate-Formats are contain the same information, for example, the second format contains the IP, too.}
%\todo{Does the new Certificate contain the IP?}

\inGerman{Der erste Teil, zwischen den Zeilen 1 und 19, ist einfach ein "offentlicher PGP-Schl"ussel.
Der zweite Teil, ab Zeile 20, enthält spezifische Zusatzinformationen für RetroShare.
Enthalten sein muss die \emph{SSLID} an Hand derer RetroShare die Installation findet und \emph{LOCATION} die die für Menschen lesbare Form der Installation darstellt.
Informationen wie die internen/externen IP- bzw. DynDNS-Adresse muss nicht unbedingt enthalten sein.
Die externe IP-Adresse ist nur eine Hilfe wenn sie noch gültig ist.
In den meisten Fällen alle 24 Stunden eine neue IP-Adresse zugewiesen und RetroShare wird diese auch selbst herausfinden.
Siehe dazu Abschnitt \ref{Verbindung zu Freunden}.
Wenn man das Ganze in eine Textdatei mit der Endung \emph{.rsc} speichtert, kann man es als Datei per eMail, Stick oder sonstiges versenden und von RetroShare importieren.}
\inEnglish{The first part of the certificate (lines 1-19) is simply a PGP public key.
The second part (lines 20-end) contains specific extra information needed by RetroShare.
The certificate must contain the \emph{SSLID} and the \emph{LOCATION}, to be able to find and connect to your installation.
The intern/extern IP resp. the (Dyn)DNS-adress is optional, RetroShare can exchange this information with your friend on your first connection.
The extern IP is only helpful, if it is still valid.
In most cases (home setup), the extern IP will change every 24h, but this doesn't matter, as RetroShare can figure this out by itself (see \ref{Verbindung zu Freunden}).
If you're exporting your key to a *.rsc file, you can mail it, put it on an USB-stick etc.}

\subsection{\inGerman{Das erste \emph{eigene} Netzwerk}
\inEnglish{The first \emph{own} network}}
%
\inGerman{Als nächstes solltest du nun Freunde hinzufügen, da ein \glqq friend to friend\grqq Netzwerk ohne Freunde keinen Sinn macht. Ich persönlich nutze immer die Methode des \glqq manuellen Austauschs der Zertifikate\grqq.}
\inEnglish{Now you should add some friends, because a F2F-network without friends is pointless.}

\inGerman{Du und dein Freund müssen nun ein einziges Mal eure Zertifikate austauschen, z.B. per eMail, USB-Stick oder sonstiges.
Jeder muß das Zertifikat des anderen hinzufügen, ansonsten könnt ihr euch nicht verbinden. 
Bei der \emph{manuellen Eingabe} des Zertifikats kann die Meldung \glqq Zertifikat korrupt\grqq erscheinen.
Dies bedeutet, dass beim Kopieren des Zertifikats etwas schief gegangen ist, z.B. mit zusätzlichen Leerzeichen oder Zeilenumbrüchen und/oder einer falschen Kodierung. 
Rechts neben dem Feld wo man das Zertifikat des Freundes hineinkopiert hat, leuchtet ein grüner Haken. 
Wenn er gedrückt wird, versucht RetroShare die Kodierung anzupassen und falsche Whitespaces zu entfernen. 
Falls danach das Zertifikat immernoch als korrupt gemeldet wird, sollte man es als Dateien austauschen.}
\inEnglish{You and your friend need to exchange your certificates. If there's a message about "corrupt certificate", one of you did something wrong with the copying (like wrong encoding, unintended extra characters etc.).
You can try the exchange with files, if the corrupt certificate message stays.}

\inGerman{Weiter solltet ihr im Optionenmenü die maximale Up- und Download-Geschwindigkeit einstellen, die ihr RetroShare erlauben wollt.
Voreingestellt sind 200KB/s down und 50KB/s up.
Wie schnell der eigene Internetanschluss ist, kann man durch einen Speedtest z.B. bei \url{http://wieistmeineip.de} herausfinden.
Wenn du zuviel Upload einstellst, sodass dein Upload komplett ausgelastet ist, wird sich z.B. das Surfen deutlich verlangsamen.
Stellt man zu wenig ein, können Freunde nur unnötig langsam von dir herunterladen.}
\inEnglish{You both should adjust your speed settings to the optimal Up- and Downloadspeed.
The default settings are 200KB/s down and 50KB/s up.
If you set the upload limit too high, it will slow down browsing, if you set it too low, your friends have to wait unnecessarily long for downloads from you.}

\inGerman{Ich persönlich habe laut Speedtest einen Upload von 150KB/s und habe RetroShare 120KB/s gegeben, sodass ich noch genug Reserve habe und RetroShare immer laufen lassen kann, solange mein Laptop läuft. 
Am besten ist es wohl, einfach etwas herumexperimentieren. 
Meine Download-Geschwindigkeit habe ich frei Schnauze auf 1000KB/s gesetzt, da ich so schnell wie möglich downloaden möchte.}
\inGerman{Personally, I've got according to a speed test an upload from 150KB/s, so I set RetroShare's upload limit to 120kB/s.
This way, I have enough reserved bandwidth and am able to let RetroShare continously.
The upload limit I've set to 1000KB/s, because I want to download as fast as possible.}

\inGerman{Weiter sollte jeder von euch seinen Router konfigurieren. 
Am einfachsten ist es, UPnP zu aktivieren und in RetroShare diese Option zu verwenden. 
Die beste Performance von RetroShare erhält man jedoch, indem man explizit einen Port (z.B. den voreingestellten Port 7812) weiterleitet. 
Eine Anleitung dafür kann ich hier nicht geben, da diese von Router zu Router unterschiedlich ist.
Eine Suche im Internet sollte hier weiterhelfen.
Es muss sowohl TCP als auch UDP weitergeleitet werden.
Überprüfen kann man den Erfolg der Port-Weiterleitung mithilfe der Seite \url{http://canyouseeme.org}.
Der Computer auf dem RetroShare laufen soll muss dazu natürlich angeschalten sein.}
\inEnglish{In the next step, each of you both should configure his router.
The easiest way is to enable UPnP and set RetroShare's server setting to use it.
The best performance you'll get, if you manually forward a port in your router (e.g. the default port 7812).
We can't give an instruction how port forwarding can be done, as it is different on each router.
A search in the internet should help you.
You have to forward both TCP and UDP.
Use e.g. the site \url{http://canyouseeme.org} to verify, if your port is forwarded.
The computer, on which RetroShare runs, must be turned on for that, of course.
}

\inGerman{Nachdem ihr alle diese Sachen beherzigt habt, solltet ihr euch nun nach einiger Zeit verbunden haben. 
Viel Spass beim Benutzen von RetroShare!}
\inEnglish{After doing all this stuff, you and your first friend should connect after some time.
Have fun using RetroShare!}

\subsection{\inGerman{Freunde finden}\inEnglish{Finding friends}} \label{schluesselaustausch}
\inGerman{Wenn du RetroShare testen willst, allerdings privat niemanden kennst, den du überzeugen kannst bzw. nicht genug, so gibt es im Internet einige Möglichkeiten, mit Fremden Schlüssel auszutauschen.
Die Gefahr dabei ist natürlich, dass man eventuell Leute hinzufügt, die man nicht möchte.}

Die mir bekannten Seiten dazu seien hier genannt:
\begin{itemize}
 \item \url{http://retroshare.sourceforge.net/forum/}: 
 \inGerman{Das offizielle Forum von RetroShare hat auch einen Schlüsselaustausch-Thread, allerdings müssen Forenposts dort erst manuell freigegeben werden, was bis zu einer Woche dauern kann, da das Forum für die Entwickler keine Priorität hat.}
 \inEnglish{The official Forum of RetroShare has a key-exchange thread. Unfortunately the forum doesn't allow posts from new users, all posts have to be checked first. Be patient, the admin isn't looking every day.}
 \item \url{http://f2f-fr.net/w2c}: 
 \inGerman{Eine französische Seite, in der man seinen Schlüssel an einen sogenannten ``Einführungsserver'' schicken kann, der einen dann automatisch als Freund bestätigt. Danach kann man über Chat Lobbys mit anderen Neulingen dort reden und sich bei gegenseitigem Interesse gegenseitig als Freunde hinzufügen.}
 \inEnglish{A french site, where you can put your key in a webform. A "Chat Server" is adding you automatically then, and will give you access to some lobbys. You can chat then with other people looking for friends and add some of them as friends, if you want.}
\end{itemize}
\inGerman{Wie viele Freunde man auf diese Art hinzufügen möchte, muss jedoch jeder Nutzer für sich selbst entscheiden.}
\inEnglish{How many such friends you want to add, is your decision.}

\subsection{\inGerman{weitere Hinweise zur Benutzung}\inEnglish{other tips and tricks}}

\inGerman{Die Oberfläche von RetroShare sollte eigentlich weitestgehend selbsterklärend sein, zumindest, wenn man dieses Handbuch durchgelesen hat. 
Leider ist sie meiner Meinung nach etwas kompliziert, und einige ``Tricks'' sollen hier extra erwähnt werden:}
\inEnglish{The user interface of RetroShare should be quite self-explanatory, at least, if you have read this manual.
Some options are a little hidden, though, and should get noted here:}

\begin{itemize}
 \item \underline{Freunde empfehlen}: 
 \inGerman{Will man RetroShare-intern einen Freund einem anderen empfehlen, so ist es am einfachsten, die eingebaute Empfehlung zu verwenden (Rechtsklick auf den Freund) oder den zugehörigen ``Zertifikatslink'' (Rechtsklick auf die einen Ort eines Freundes) zu schicken. Ein Zertifikatslink hat die Form

  retroshare://certificate?sslid=aa61180732ee9051aa61180732ee9051\&{\color{red}gpgid=A1047F82\&gpgbase64=mQENBE9aG9gBCADAA09oJZJzUSU7n1udB7o34L9orvp5ckZkB0I/yq4V9446mOiorSfaVDQnFjETV0NbrM51RIvenUHP75Jzq4/QXcyFE5B+poMg3pVrVUXAm31HZd2StRxprxVFpPTplLutNv79WN+NSih8fBFjc8buAKquff6uue9tvX0mj47NNzp6iSR6+Ae1vzEI0IIw+wKxsdsMcZdPjbJXYRD5eaYnxbhjdJcpr3Sx73XTbOONs7DoNyxrgEURfV4PIVUWF/zmZmKgp5Gtko4k4k+LRKaTNHZ7rEvPvDqv8O2lXrAmdXG81kuCXH6K4/ZrLQzltvhqWMzmxgD9OuJunRv2X2bjABEBAAG0M2tsYXVzIChnZW5lcmF0ZWQgYnkgUmV0cm9zaGFyZSkgPGtsYXVzQGV4YW1wbGUuY29tPokBOAQTAQIAIgUCT1ob2AIbLwYLCQgHAwIGFQgCCQoLBBYCAwECHgECF4AACgkQnDxC1maIY6nF8Qf/ZZZPni5VYKWKaN+j5rIkjWoDtySeE3iCOCAJtyjiXVGsRWKaSFPZSiL+8VZl6OPYN6oAJnWjyDDf2Ql/QUJKfKpcqpe0SowxMGuDiOHkwCp+Ac6g1tWAi+zRPwP93Af1nI7dNa1TZGmjVJSIvU90JTUM7gCx7vpJf59UZqDatggLxzaeNo2ryXjD+/npRCqeF/kKRPp3/Oac1IIsYU8JK37uzIQJv3Nv3yBkP73OoZhEq7+g2tw9xsFfZn8skbpReXutwvuVFJbGGJLCWikN/DgGl21RaobeiFt109T6LMbsBkIHq9paPVbu7yTgL2TUiLka9sOAwIjZX0tM5DpzIg\&gpgchecksum=lrVI}\&{\color{green}location=laptop\&name=test\&
locipp=192.168.0.199:39270;\&extipp=123.142.101.16:39270;}

  In diesem langen Link steckt das u.a. das gesamte {\color{red}Zertifikat} und die aktuelle {\color{green}IP-Adresse}, dadurch wird der Schlüsselaustausch zu einem einzigen Klick.}
  \inEnglish{If you want to recommend RetroShare-intern one friend to another, you should use the "certificate link". A certificate link has the following format

  retroshare://certificate?sslid=aa61180732ee9051aa61180732ee9051\&{\color{red}gpgid=A1047F82\&gpgbase64=mQENBE9aG9gBCADAA09oJZJzUSU7n1udB7o34L9orvp5ckZkB0I/yq4V9446mOiorSfaVDQnFjETV0NbrM51RIvenUHP75Jzq4/QXcyFE5B+poMg3pVrVUXAm31HZd2StRxprxVFpPTplLutNv79WN+NSih8fBFjc8buAKquff6uue9tvX0mj47NNzp6iSR6+Ae1vzEI0IIw+wKxsdsMcZdPjbJXYRD5eaYnxbhjdJcpr3Sx73XTbOONs7DoNyxrgEURfV4PIVUWF/zmZmKgp5Gtko4k4k+LRKaTNHZ7rEvPvDqv8O2lXrAmdXG81kuCXH6K4/ZrLQzltvhqWMzmxgD9OuJunRv2X2bjABEBAAG0M2tsYXVzIChnZW5lcmF0ZWQgYnkgUmV0cm9zaGFyZSkgPGtsYXVzQGV4YW1wbGUuY29tPokBOAQTAQIAIgUCT1ob2AIbLwYLCQgHAwIGFQgCCQoLBBYCAwECHgECF4AACgkQnDxC1maIY6nF8Qf/ZZZPni5VYKWKaN+j5rIkjWoDtySeE3iCOCAJtyjiXVGsRWKaSFPZSiL+8VZl6OPYN6oAJnWjyDDf2Ql/QUJKfKpcqpe0SowxMGuDiOHkwCp+Ac6g1tWAi+zRPwP93Af1nI7dNa1TZGmjVJSIvU90JTUM7gCx7vpJf59UZqDatggLxzaeNo2ryXjD+/npRCqeF/kKRPp3/Oac1IIsYU8JK37uzIQJv3Nv3yBkP73OoZhEq7+g2tw9xsFfZn8skbpReXutwvuVFJbGGJLCWikN/DgGl21RaobeiFt109T6LMbsBkIHq9paPVbu7yTgL2TUiLka9sOAwIjZX0tM5DpzIg\&gpgchecksum=lrVI}\&{\color{green}location=laptop\&name=test\&
locipp=192.168.0.199:39270;\&extipp=123.142.101.16:39270;}
  
  In this long single link, all information needed by RetroShare is stored: the {\color{red} certificate}, {\color{green} current IP address}. This makes exchanging friends to a simple mouse click.}
 \item \underline{rscollection}: 
 \inGerman{Dies ist im Prinzip nichts anderes als eine kleine XML Datei, die eine Liste von Ordnern und/oder Dateien enthält. Diese Datei kann in RetroShare im Fenster ``Transfers'' mit dem Button ``Öffne Kollektion'' geöffnet werden und die eigentlichen Dateien können ausgewählt und heruntergeladen werden. Siehe dazu auch den Abschnitt \ref{rscollection}.}
 \inEnglish{}
\end{itemize}
\todo{more tips}