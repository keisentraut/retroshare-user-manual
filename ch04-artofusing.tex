\section{\inGerman{Benutzen von RetroShare}\inEnglish{Using RetroShare}}
\subsection{Der erste Start}
Beim ersten Start wird RetroShare dir die Möglichkeit bieten, einen neuen GnuPG-Schlüsselpaar zu erzeugen. Falls du bereits einen GnuPG-Schlüssel hast, z.B. für Email-Verschlüsselung, kannst du auch diesen verwenden. Dabei muß erwähnt werden das nur \emph{RSA}-Schlüssel verwendbar sind. Mit \emph{ElGamal} oder \emph{DSA} kann zwar eine SSL-ID erstellt werden, nach einem Neustart von RetroShare ist es allerdings nicht in der Lage es zu entschlüsseln.

Wenn RetroShare ein GnuPG-Schlüsselpaar erstellt, benötigt es einen (Nick-)Namen, eine beliebige Email-Adresse (wird nicht auf Gültigkeit geprüft und auch später nicht benötigt), ein Passwort und einen sogenannten \glqq Ort\grqq . Bei der eMail-Adresse steht zwar \emph{optional}, aber RetroShare kann ohne eine Adresse keinen Schlüssel von GnuPG erstellen lassen.

Der Ort dient dazu, das Menschen z.B. einen Laptop und einen PC unterscheiden können. Denn \glqq Laptop\grqq oder \glqq Server\grqq lassen sich leichter unterscheiden als rein zufällige Zahlenkollonen wie\\ \glqq 2ddf5cefd2517fb41fd46b5bbb7ce36d\grqq. RetroShare generiert mit Hilfe des GnuPG-Key für jede Installation einen neuen SSL-Key, anhand der die einzelnen Installationen unterschieden werden und die Kommunikation verschlüsselt wird. Bei sämtlichen Installationen kann man denselben GnuPG-Key verwenden.

Ein RetroShare-Zertifikat sieht folgendermaßen aus (dieses hier ist kein real benutztes):
\begin{lstlisting}
 -----BEGIN PGP PUBLIC KEY BLOCK-----
Version: GnuPG v1.4.11 (GNU/Linux)

mQENBE9aG9gBCADAA09oJZJzUSU7n1udB7o34L9orvp5ckZkB0I/yq4V9446mOio
rSfaVDQnFjETV0NbrM51RIvenUHP75Jzq4/QXcyFE5B+poMg3pVrVUXAm31HZd2S
tRxprxVFpPTplLutNv79WN+NSih8fBFjc8buAKquff6uue9tvX0mj47NNzp6iSR6
+Ae1vzEI0IIw+wKxsdsMcZdPjbJXYRD5eaYnxbhjdJcpr3Sx73XTbOONs7DoNyxr
gEURfV4PIVUWF/zmZmKgp5Gtko4k4k+LRKaTNHZ7rEvPvDqv8O2lXrAmdXG81kuC
XH6K4/ZrLQzltvhqWMzmxgD9OuJunRv2X2bjABEBAAG0M2tsYXVzIChnZW5lcmF0
ZWQgYnkgUmV0cm9zaGFyZSkgPGtsYXVzQGV4YW1wbGUuY29tPokBOAQTAQIAIgUC
T1ob2AIbLwYLCQgHAwIGFQgCCQoLBBYCAwECHgECF4AACgkQnDxC1maIY6nF8Qf/
ZZZPni5VYKWKaN+j5rIkjWoDtySeE3iCOCAJtyjiXVGsRWKaSFPZSiL+8VZl6OPY
N6oAJnWjyDDf2Ql/QUJKfKpcqpe0SowxMGuDiOHkwCp+Ac6g1tWAi+zRPwP93Af1
nI7dNa1TZGmjVJSIvU90JTUM7gCx7vpJf59UZqDatggLxzaeNo2ryXjD+/npRCqe
F/kKRPp3/Oac1IIsYU8JK37uzIQJv3Nv3yBkP73OoZhEq7+g2tw9xsFfZn8skbpR
eXutwvuVFJbGGJLCWikN/DgGl21RaobeiFt109T6LMbsBkIHq9paPVbu7yTgL2TU
iLka9sOAwIjZX0tM5DpzIg==
=7I5x
-----END PGP PUBLIC KEY BLOCK-----
--SSLID--2ddf5cefd2517fb41fd46b5bbb7ce36d;--LOCATION--laptop;
--LOCAL--192.168.2.103:23822;--EXT--93.61.21.14:7812;
--DYNDNS--<subdomain>.<domain>.com;
\end{lstlisting}

Der erste Teil, zwischen den Zeilen 1 und 19, ist einfach ein GnuPG-Zertifikat. Der zweite Teil, ab Zeile 20, enthält spezifische Zusatzinformationen für RetroShare. Enthalten sein muss die \emph{SSLID} an Hand derer RetroShare die Installation findet und \emph{LOCATION} die die für Menschen lesbare Form der Installation darstellt. Informationen wie die internen/externen IP- bzw. DynDNS-Adresse muss nicht unbedingt enthalten sein. Die externe IP-Adresse ist nur eine Hilfe wenn sie noch gültig ist. In den meisten Fällen alle 24 Stunden eine neue IP-Adresse zugewiesen und RetroShare wird diese auch selbst herausfinden. Siehe dazu Abschnitt \ref{Verbindung zu Freunden}. Wenn man das Ganze in eine Textdatei mit der Endung \emph{.rsc} speichtert, kann man es als Datei per eMail, Stick oder sonstiges versenden und von RetroShare importieren.

\subsection{Das erste \emph{eigene} Netzwerk}

Als nächstes solltest du nun Freunde hinzufügen, da ein \glqq friend to friend\grqq Netzwerk ohne Freunde keinen Sinn macht. Ich persönlich nutze immer die Methode des \glqq manuellen Austauschs der Zertifikate\grqq.

Du und dein Freund müssen nun ein einziges Mal eure Zertifikate austauschen, z.B. per eMail, USB-Stick oder sonstiges. Jeder muß das Zertifikat des anderen hinzufügen, ansonsten könnt ihr euch nicht verbinden. Bei der \emph{manuellen Eingabe} des Zertifikats kann die Meldung \glqq Zertifikat korrupt\grqq erscheinen. Dies bedeutet, dass beim Kopieren des Zertifikats etwas schief gegangen ist, z.B. mit zusätzlichen Leerzeichen oder Zeilenumbrüchen und/oder einer falschen Kodierung. Rechts neben dem Feld wo man das Zertifikat des Freundes hineinkopiert hat, leuchtet ein grüner Haken. Wenn er gedrückt wird, versucht RetroShare die Kodierung anzupassen und falsche Whitespaces zu entfernen. Falls danach das Zertifikat immernoch als korrupt gemeldet wird, sollte man es als Dateien austauschen.

Weiter solltet ihr im Optionenmenü die maximale Up- und Download-Geschwindigkeit einstellen, die ihr RetroShare erlauben wollt. Voreingestellt sind 200KB/s down und 50KB/s up. Wie schnell der eigene Internetanschluss ist, kann man durch einen Speedtest z.B. bei \url{http://wieistmeineip.de} herausfinden. Wenn du zuviel Upload einstellst, sodass dein Upload komplett ausgelastet ist, wird sich z.B. das Surfen deutlich verlangsamen. Stellt man zu wenig ein, können Freunde nur unnötig langsam von dir herunterladen.

Ich persönlich habe laut Speedtest einen Upload von 150KB/s und habe RetroShare 130KB/s gegeben, sodass ich noch genug Reserve habe und RetroShare immer laufen lassen kann, solange mein Laptop läuft. 
Am besten ist es wohl, einfach etwas herumexperimentieren. 
Meine Download-Geschwindigkeit habe ich frei Schnauze auf 1000KB/s gesetzt, da ich so schnell wie möglich downloaden möchte.

Weiter sollte jeder von euch seinen Router konfigurieren. 
Am einfachsten ist es, UPnP zu aktivieren und in RetroShare diese Option zu verwenden. 
Die beste Performance von RetroShare erhält man jedoch, indem man explizit einen Port (z.B. den voreingestellten Port 7812) weiterleitet. 
Eine Anleitung dafür kann ich hier nicht geben, da diese von Router zu Router unterschiedlich ist.
Eine Suche im Internet sollte hier weiterhelfen.
Es muss sowohl TCP als auch UDP weitergeleitet werden.
Überprüfen kann man den Erfolg der Port-Weiterleitung mithilfe der Seite \url{http://canyouseeme.org}.
Der Computer auf dem RetroShare laufen soll muss dazu natürlich angeschalten sein.

Nachdem ihr alle diese Sachen beherzigt habt, solltet ihr euch nun nach einiger Zeit verbunden haben. Viel Spass beim Benutzen von RetroShare!

\subsection{Freunde finden} \label{schluesselaustausch}
Wenn du RetroShare testen willst, allerdings privat niemanden kennst, den du überzeugen kannst bzw. nicht genug, so gibt es im Internet einige Möglichkeiten, mit Fremden Schlüssel auszutauschen.
Die Gefahr dabei ist natürlich, dass man eventuell Leute hinzufügt, die man nicht möchte.

Die mir bekannten Seiten dazu seien hier genannt:
\begin{itemize}
 \item \url{http://retroshare.sourceforge.net/forum/}: Das offizielle Forum von RetroShare hat auch einen Schlüsselaustausch-Thread, allerdings müssen Forenposts dort erst manuell freigegeben werden, was bis zu einer Woche dauern kann, da das Forum für die Entwickler keine Priorität hat. Somit macht es hier nicht wirklich viel Spass.
 \item \url{http://retroshare.info}: Ein (inoffizielles) deutsches RetroShare-Forum, dass für Neulinge auch einen Thread für Schlüsselaustausch hat. Dies ist die wohl beste Methode zu deutschsprachigen Freunden zu kommen.
 \item \url{http://f2f-fr.net/w2c}: Eine französische Seite, in der man seinen Schlüssel an einen sogenannten ``Einführungsserver'' schicken kann, der einen dann automatisch als Freund bestätigt. Danach kann man über Chat Lobbys mit anderen Neulingen dort reden und sich bei gegenseitigem Interesse gegenseitig als Freunde hinzufügen. Auf dieser Seite erhält man viele französische Freunde.
\end{itemize}
Wie viele Freunde man auf diese Art hinzufügen möchte, muss jedoch jeder Nutzer für sich selbst entscheiden.

\subsection{weitere Hinweise zur Benutzung}

Die Oberfläche von RetroShare sollte eigentlich weitestgehend selbsterklärend sein, zumindest, wenn man dieses Handbuch durchgelesen hat. 
Leider ist sie meiner Meinung nach etwas kompliziert, und einige ``Tricks'' sollen hier extra erwähnt werden:

\begin{itemize}
 \item \underline{Freunde empfehlen}: Will man RetroShare-intern einen Freund einem anderen empfehlen, so ist es am einfachsten, die eingebaute Empfehlung zu verwenden (Rechtsklick auf den Freund) oder den zugehörigen ``Zertifikatslink'' (Rechtsklick auf die einen Ort eines Freundes) zu schicken. Ein Zertifikatslink hat die Form

  retroshare://certificate?sslid=aa61180732ee9051aa61180732ee9051\&{\color{red}gpgid=A1047F82\&gpgbase64=mQENBE9aG9gBCADAA09oJZJzUSU7n1udB7o34L9orvp5ckZkB0I/yq4V9446mOiorSfaVDQnFjETV0NbrM51RIvenUHP75Jzq4/QXcyFE5B+poMg3pVrVUXAm31HZd2StRxprxVFpPTplLutNv79WN+NSih8fBFjc8buAKquff6uue9tvX0mj47NNzp6iSR6+Ae1vzEI0IIw+wKxsdsMcZdPjbJXYRD5eaYnxbhjdJcpr3Sx73XTbOONs7DoNyxrgEURfV4PIVUWF/zmZmKgp5Gtko4k4k+LRKaTNHZ7rEvPvDqv8O2lXrAmdXG81kuCXH6K4/ZrLQzltvhqWMzmxgD9OuJunRv2X2bjABEBAAG0M2tsYXVzIChnZW5lcmF0ZWQgYnkgUmV0cm9zaGFyZSkgPGtsYXVzQGV4YW1wbGUuY29tPokBOAQTAQIAIgUCT1ob2AIbLwYLCQgHAwIGFQgCCQoLBBYCAwECHgECF4AACgkQnDxC1maIY6nF8Qf/ZZZPni5VYKWKaN+j5rIkjWoDtySeE3iCOCAJtyjiXVGsRWKaSFPZSiL+8VZl6OPYN6oAJnWjyDDf2Ql/QUJKfKpcqpe0SowxMGuDiOHkwCp+Ac6g1tWAi+zRPwP93Af1nI7dNa1TZGmjVJSIvU90JTUM7gCx7vpJf59UZqDatggLxzaeNo2ryXjD+/npRCqeF/kKRPp3/Oac1IIsYU8JK37uzIQJv3Nv3yBkP73OoZhEq7+g2tw9xsFfZn8skbpReXutwvuVFJbGGJLCWikN/DgGl21RaobeiFt109T6LMbsBkIHq9paPVbu7yTgL2TUiLka9sOAwIjZX0tM5DpzIg\&gpgchecksum=lrVI}\&{\color{green}location=laptop\&name=test\&
locipp=192.168.0.199:39270;\&extipp=123.142.101.16:39270;}

  In diesem langen Link steckt das u.a. das gesamte {\color{red}Zertifikat} und die aktuelle {\color{green}IP-Adresse}, dadurch wird der Schlüsselaustausch zu einem einzigen Klick.
 \item \underline{rscollection}: Dies ist im Prinzip nichts anderes als eine kleine XML Datei, die eine Liste von Ordnern und/oder Dateien enthält. Diese Datei kann in RetroShare im Fenster ``Transfers'' mit dem Button ``Öffne Kollektion'' geöffnet werden und die eigentlichen Dateien können ausgewählt und heruntergeladen werden. Siehe dazu auch den Abschnitt \ref{rscollection}.
 \item TODO: mehr
\end{itemize}